\documentclass[11pt]{article}
\usepackage[utf8]{inputenc}
\usepackage{graphicx}
\usepackage{amsmath}
\usepackage{hyperref}
\usepackage{geometry}
\usepackage{setspace}
\usepackage{titling}
\usepackage{authblk}
\usepackage{cite}

\geometry{margin=1in}
\setstretch{1.2}

\title{\bfseries\Large \textit{CrossSpec}: Leveraging Cross-Species Data Integration from Homologous and Analogous Structures for Layer Segmentation in Kidney Pathology}

\author[1]{Fardeen Bablu}

\affil[1]{Department of Computer Science, Vanderbilt University, Nashville, TN, USA \\ \texttt{fardeen.bablu@vanderbilt.edu}}

\date{\today}

\begin{document}

\maketitle

\begin{abstract}
\noindent
This report investigates the use of cross-species data integration to enhance layer segmentation for kidney pathology. Building on prior work~\cite{zhu2025crossspeciesdataintegrationenhanced}, we explore incorporating both homologous and analogous structures using public datasets such as NuInsSeg~\cite{mahbod2023nuinsseg}. This approach aims to improve model generalization and segmentation performance, especially in scenarios with limited clinical data.
\end{abstract}

\textbf{Keywords}: Layer Segmentation, Kidney Pathology, Cross-Species Data Integration

\section*{1. Introduction}
\noindent
Robust segmentation of histological images is critical in kidney pathology research. Cross-species data integration through using both homologous and analogous structures offers a promising avenue to enhance segmentation models. Homologous structures are anatomical features inherited from a common ancestor. While they may serve different functions, they share underlying structural similarities. In contrast, analogous structures perform similar functions but evolved independently and do not share a common evolutionary origin. Building on prior work~\cite{zhu2025crossspeciesdataintegrationenhanced}, which focused on homologous structures, this study extends the exploration to include analogous structures as well. We hypothesize that combining both forms of biological similarity can further improve segmentation outcomes. We utilize public datasets such as NuInsSeg~\cite{mahbod2023nuinsseg} to implement this approach. Our goal is to enhance model generalization and performance, particularly in settings where annotated clinical data is scarce.

\section*{2. Method}
\noindent
Your method description goes here.

\bibliographystyle{plain}
\bibliography{references}

\end{document}
