\documentclass[11pt]{article}
\usepackage[utf8]{inputenc}
\usepackage{graphicx}
\usepackage{amsmath}
\usepackage{hyperref}
\usepackage{geometry}
\usepackage{setspace}
\usepackage{titling}
\usepackage{authblk}
\usepackage{cite}

\geometry{margin=1in}
\setstretch{1.2}

\title{\bfseries\Large \textit{CrossSpec}: Using Cross-Species Data Integration from Homologous and Analogous Structures in Layer Segmentation for Kidney Pathology}

\author[1]{Fardeen Bablu}

\affil[1]{Department of Computer Science, Vanderbilt University, Nashville, TN, USA \\ \texttt{fardeen.bablu@vanderbilt.edu}}

\date{\today}

\begin{document}

\maketitle

\begin{abstract}
\noindent
This report explores the use of cross-species data integration in layer segmentation for kidney pathology. We leverage homologous and analogous structures to improve segmentation performance using public datasets such as NuInsSeg~\cite{mahbod2023nuinsseg}. This approach has potential to improve model generalization and performance, especially under conditions of limited clinical data availability.
\end{abstract}

\textbf{Keywords}: Layer Segmentation, Kidney Pathology, Cross-Species Data Integration

\section*{1. Introduction}
\noindent
Kidney pathology research increasingly relies on robust segmentation of histological images. Cross-species integration of homologous and analogous structures presents an opportunity to enhance segmentation models. In this work, we explore approaches that leverage such integration, building upon datasets such as NuInsSeg~\cite{mahbod2023nuinsseg}.

\section*{2. Method}
\noindent
Your method description goes here.


\bibliographystyle{plain}
\bibliography{references}

\end{document}
